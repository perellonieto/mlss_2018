\chapter{Causality by \href{https://staff.fnwi.uva.nl/j.m.mooij/}{Joris Mooij}}

There are many questions in science that are casual (eg. climatology,
healthcare, \dots).

\section{Introduction}


Causation is not correlation (gives example with chocolate consumption being
correlated with number of nobel prizes in different countries, while there is
no actual correlation between both).

In order to represent causal relations we can use \emph{causal graphs}
(directed graphs) in wich nodes are variables $X_n$ from (a vocabulary?) $V$.
While directed edges indicate that the first variable causes direclty another
variable respect to $V$. An example of a cyclic graph where every adjacent
variable is directly connected to its neighbours is the encoding of standing
domino pieces.

It is possible to modify the \emph{causal graph} with an \emph{intervention},
in the example the domino piece $X_2$ is glued to the floor, removing the
direct connections from other pieces to $X_2$, but possibly keeping a
connection from $X_2$ to the adjacent (in case an external user can still force
$X_2$ to fall).

A \emph{perfect (``surgical'', ``atomic'') intervention} on a set of variables
$X \subseteq V$, is an external enforced change of the system (eg. the
previous example).

A \emph{confounder} is a latent common cause:

\begin{mybox}
  $H$ is a confounder of $X$ and $Y$ if:
  \begin{enumerate}
    \item $H$ causes $X$ directly w.r.t. $\{X, Y, H\}$
    \item $H$ causes $Y$ directly w.r.t. $\{X, Y, H\}$
  \end{enumerate}
\end{mybox}

We will denote latent confounders by \emph{bidirected edges} in a causal graph.

\section{Defining causality in terms of probabilities}

\emph{Simpson's paradox} shows that if we interpret the probabilities as causes
we may make wrong decisions. As an example, shows the recovery rate of a drug
test in which depending on the groups separation the most probable outcome
changes.

\begin{mybox}
\begin{enumerate}
  \item The probability of recovery is higher for patients that took the drug
    \begin{align}
      p(\text{recovery}|\text{drug}) > p(\text{recovery}|\text{no drug})
    \end{align}
  \item For both male and female patients the relation was oposite
    \begin{align}
      p(\text{recovery}|\text{drug, male}) <& p(\text{recovery}|\text{no drug,
      male}) \\
      p(\text{recovery}|\text{drug, female}) <& p(\text{recovery}|\text{no drug,
      female})
    \end{align}
\end{enumerate}
\end{mybox}

\emph{endogenous variables} are binary variables that we are interested in.

\emph{Exogenous variables} are latent, independent binary variables that affect
externally the state of our endogeneous variables.

A \emph{Structureal Causal Model (SCM)}, also knwon as \emph{Structural
Equation Model (SEM)}, is a tuple \dots

There is one Structural equations  per endogenous variable.

\begin{enumerate}
  \item a product of standard measures \dots
  \item a product of standard measures \dots
  \item Measurable mapping \dots
  \item A product probbility measure\dots
\end{enumerate}

An augmented functional graph $G^a(M)$ depicts the exogenous variables while
the functional graph $G(M)$ doesn't.

\begin{mybox}
  If $M$ has no \emph{self-loops}, the causal graph of $M$ is a subgraph of the
  functional graph $G(M)$.
\end{mybox}

\begin{definition}
We call the family of sets of probability distributions of the solutions of
$M_{do(l,E_l)}$ \dots
\end{definition}

some of the previous points are teh basic difference between causal models and
probabilistic models.

We donote a marginalization of the model $M$ with respect two variables $X_2$
and $X_4$ as $M^{\backslash\{2,4\}}$.

See the following extra references \cite{de2013global}, and
\cite{bongers2018random}, \cite{blom2018generalized},
Bongers et al., 2018.


\begin{definition}
  Definitions of: Independence and conditional independence
\end{definition}

\begin{definition}
  Definition of: nodes blocking a path
\end{definition}

\begin{theorem}
  For an \emph{acyclic} SCM, \dots
\end{theorem}

\emph{Reichenbach's} principle of common cause, the dependence $X | Y$

The Reichenbach's Principe may fail in case of \emph{selection bias} (related
with the explaining away problem)


\section{Causal Inference: Predicting Causal Effects}

\begin{theorem}
  Back-Door Criterion \cite{pearl2000causal}
\end{theorem}

\section{Resolving Simpson's paradox}

It is important to realize that ``seing'' is not the same as ``doing''.

\begin{itemize}
  \item $p(R = 1|D = 1)$: the probability that somebody recovers, given the
    observation that the person took the drug.
  \item $p(R = 1| \text{do}(D=1))$: the probability that somebody recovers, if
    we force the person to take the drug.
\end{itemize}

In practice randomized control trial

\begin{mybox}
  A \textbf{randomized controlled trial} (or randomized control trial;[2] RCT) is a type
of scientific (often medical) experiment which aims to reduce bias when testing
a new treatment. The people participating in the trial are randomly allocated
to either the group receiving the treatment under investigation or to a group
receiving standard treatment (or placebo treatment) as the control.
Randomization minimises selection bias and the different comparison groups
allow the researchers to determine any effects of the treatment when compared
with the no treatment (control) group, while other variables are kept constant.
The RCT is often considered the gold standard for a clinical trial. RCTs are
often used to test the efficacy or effectiveness of various types of medical
intervention and may provide information about adverse effects, such as drug
reactions. Random assignment of intervention is done after subjects have been
assessed for eligibility and recruited, but before the intervention to be
studied begins. -- \textbf{Wikipedia}
\end{mybox}

\section{Causal Discovery: from data to causal graph}

Randomized controlled trials \cite{fisher1935design} are one solution to avoid
previously seen problems.

\begin{enumerate}
  \item Divide patients into two groups: treatment and control randomly
  \item Patients with the treatment group are forced to take a drug, and
    patients in the group are forced to not take the drug (but to take a
    placebo instead): $D = C$
  \item Estimating the causal effect of the drug now becomes a standard \dots
  \item \dots
\end{enumerate}

\subsection{Local Causal Discovery (LCD)}

Simple method that Joris Mooij likes

\section{Practical aplication}

It is possible to apply $k$-fold Cross-validation to the observational data and
interventional data in order to estimate the test performance.

\section{Conclusions}

Additional readings: Causality: Models reasoning and inference, pearl 2000.
Constraint-based causal discovery for non-linear

\begin{itemize}
  \item Elements of causal inference, foundations and learning algorithms by
    Peters, Janzing, Scholkopf 2017 %\cite{peters2017elements}
  \item Causation, prediction and search by spirtes, glymour, scheines 2000
    %\cite{spirtes2000causation}
  \item correlation and causation by Wrights 1921
  \item Causal inference in statistics: an overview by Pearl 2009
  \item Simpson's paradox: an anatomy by Pearl 1999 %\cite{pearl1999simpson}
  \item Causality: models, reasoning and inference by Pearl 2000
    %\cite{pearl2003causality}
  \item Theoretical aspects of cyclic structural causal models by
    Bongers, Peters, Scholkopf, Mooij 2018 %\cite{bongers1611theoretical}
  \item Markov properties for graphical models with cycles and latent
    variables by Forré, and Mooij 2017 %\cite{forre2017markov}
\end{itemize}

Some possible aplications of Causal inference could be:

\begin{itemize}
  \item Transfer learning
  \item Domain adaptation
  \item Reinforcement learning
\end{itemize}

What tools or framewors: \textbf{there are no tools yet}, R package for
individual methods. Literature and the implementations are scattered,
\textbf{necessary to unify!}.
\bibliographystyle{apalike}
\bibliography{references}

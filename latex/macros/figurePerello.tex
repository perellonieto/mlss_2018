% This file contains macros that can be called up from connected TeX files
% It helps to summarise repeated code

%-----------------------------------------------------------------------------
% Figures with caption and description
%-----------------------------------------------------------------------------
\newcommand{\figuremacro}[3]{
% insert a centered figure with caption and description
% Parameters
%   #1: <string> filename
%   #2: <string> caption
%   #3: <string> description
%   #4: <string> label
%
    \begin{figure}[t]
    \centering
    \includegraphics[width=1\linewidth]{#1}
    \caption[#2]{\textbf{#2} #3}
    \label{#1}
    \end{figure}
}

\newcommand{\figuremacroW}[4]{
% insert a centered figure with caption, description and given width
% Parameters
%   #1: <string> filename
%   #2: <string> caption
%   #3: <string> description
%   #4: <float> width
%
	\begin{figure}[htbp]
		\centering
		\includegraphics[width=#4\linewidth]{#1}
		\caption[#2]{\textbf{#2} - #3}
		\label{#1}
	\end{figure}
}

\newcommand{\figuremacroPos}[4]{
% insert a centered figure with caption, description in given position
% Parameters
%   #1: <string> filename
%   #2: <string> caption
%   #3: <string> description
%   #4: <float> pos (e.g. htbp)
%
    \begin{figure}[#4]
    \centering
    \includegraphics[width=.5\linewidth]{#1}
    \caption[#2]{\textbf{#2} #3}
    \label{#1}
    \end{figure}
}


\newcommand{\figuremacroAll}[3]{
% insert a centered figure with caption, description in full page size
% Parameters
%   #1: <string> filename
%   #2: <string> caption
%   #3: <string> description
%
    \begin{figure*}[!t]
    \centering
    \includegraphics[width=1\textwidth]{#1}
    \caption[#2]{\textbf{#2} #3}
    \label{#1}
    \end{figure*}
}

\newcommand{\figuremacroWR}[5]{
% insert a centered figure with caption, description and given width rotated
% Parameters
%   #1: <string> filename
%   #2: <string> caption
%   #3: <string> description
%   #4: <float> width
%   #5: <float> angle
%
	\begin{figure}[htbp]
		\centering
		\includegraphics[width=#4\linewidth, angle=#5]{#1}
		\caption[#2]{\textbf{#2} - #3}
		\label{#1}
	\end{figure}
}

\newcommand{\figuremacroAllWR}[5]{
% insert a centered figure with caption, description in full page with
% particular width and rotated
% Parameters
%   #1: <string> filename
%   #2: <string> caption
%   #3: <string> description
%   #4: <float> width
%   #5: <float> angle
%
	\begin{figure*}[htbp]
		\centering
		\includegraphics[width=#4\textwidth, angle=#5]{#1}
		\caption[#2]{\textbf{#2} - #3}
		\label{#1}
	\end{figure*}
}

%% Example of use of subfiguremacro:
%\begin{figure}
%  \subfiguremacro{figure_1}{}{.5}
%  \subfiguremacro{figure_2}{}{.5}
%  \caption{}
%\end{figure}
\newcommand{\subfiguremacro}[3] {
    \begin{subfigure}[h]{#3\linewidth}
        \centering
        \includegraphics[width=1\linewidth]{#1}
        \caption{#2}
        \label{#1}
    \end{subfigure}
}

%-----------------------------------------------------------------------------
% Figures without caption (common for Beamer slides)
%-----------------------------------------------------------------------------
\newcommand{\figuremacroS}[1]{
% insert a centered figure
% Parameters
%   #1: <string> filename
%
    \begin{figure}[t]
    \centering
    \includegraphics[width=1\linewidth]{#1}
    \end{figure}
}

\newcommand{\figuremacroSW}[2]{
% insert a centered figure of specified width
% Parameters
%   #1: <string> filename
%   #2: <float> width
%
    \begin{figure}[t]
    \centering
    \includegraphics[width=#2\linewidth]{#1}
    \end{figure}
}

\newcommand{\figuremacroSWXY}[4]{
% insert a centered figure of specified width on position x,y
% Parameters
%   #1: <string> filename
%   #2: <float> width
%   #3: <float> x coordinate
%   #4: <float> y coordinate
%
\begin{textblock*}{#2\textwidth}(#3\textwidth,#4\textheight)
	\begin{figure}
		\centering
        \includegraphics[width=1.0\linewidth]{#1}
	\end{figure}
\end{textblock*}
}

\newcommand{\figuremacroSWXYstep}[5]{
% insert a centered figure of specified width on position x,y from given step
% Parameters
%   #1: <string> filename
%   #2: <float> width
%   #3: <float> x coordinate
%   #4: <float> y coordinate
%   #5: <string> steps when to show the figure
%
\begin{textblock*}{#2\textwidth}(#3\textwidth,#4\textheight)
	\begin{figure}
		\centering
        \includegraphics[width=1.0\linewidth]<#5>{#1}
	\end{figure}
\end{textblock*}
}

\newcommand{\posterfigure}[2]{
    \begin{tikzfigure}
    \includegraphics[width=#2\linewidth]{#1}
    \end{tikzfigure}
}
